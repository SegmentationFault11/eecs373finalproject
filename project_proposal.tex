% ********************************* HEADERS ***********************************
\documentclass{article}
\usepackage[top=.75in, bottom=.75in, left=.50in,right=.50in]{geometry}
\usepackage{fancyhdr}
\usepackage{titling}
\pagestyle{fancy}
\lhead{EECS 373 Project Proposal - Mega Wheels}
\rhead{\thepage}
\usepackage{color}
\usepackage{graphicx}
\usepackage{listings}
\lstset{
  % language=C++,
  showstringspaces=false,
  basicstyle={\small\ttfamily},
  numberstyle=\tiny\color{gray},
  keywordstyle=\color{blue},
  commentstyle=\color{dkgreen},
  stringstyle=\color{dkgreen},
}
\usepackage[colorlinks,urlcolor={blue}]{hyperref}
%\setlength{\parskip}{1em}
\usepackage{parskip}
\usepackage{indentfirst}
\setlength{\droptitle}{-4em}
% \usepackage{concmath}
% \usepackage[T1]{fontenc}
% ********************************* HEADERS ***********************************

\begin{document}
\title{\textbf{EECS 373 Project Proposal - Mega Wheels}}
\author{
  Sagar, Aaryaman\\
  \texttt{aary@umich.edu}
  \and
  Ma, Chong\\
  \texttt{machong@umich.edu}
  \and 
  Markov, Anton\\
  \texttt{aamarkov@umich.edu}
  \and
  Dudkov, Artem\\
  \texttt{dudkoff@umich.edu}
}
\date{}
\maketitle

\section{High Level Description} Our idea is to build a real life car battle
game in which players get to control remote control cars that we buy off the
market and customize so that they work in an embedded environment.  These cars
will be customizable before the start of the game to have targeted abilities
towards either attack, defense, speed or any other factors to be determined. 

The project will consist of several RC cars with SmartFusion Boards built onto
them that can read in input from a controller and output to control the motion
of the cars.  We will refer to these as the slave boards from now on.  There
will also be a master "server" setup with a computer and a SmartFusion Board
together.  This master will be constantly reading input from the current state
of the game, outputting results onto the computer and also giving the cars
extra abilities and enhanced battling skills.

What sets this game apart from other RC controlled car games is that each car
will have customizable powers.  Before the game starts the players will be
able to select what sort of car they want in software and that will determine
the abilities of the car and how they progress to improve throughout the game.
For example a car can be the speed car, and throughout the game as the car
kills its opponents, it gains more speed, and another car can be a sharp
shooter that shoots more accurately than other cars and gets better and better
over time as it gets more kills.  Further the car that kills a stronger car
gets more points; this encourages teaming up to destroy the dominating car. 

The master server will keep track of the current state of the game at all
times and display the results in a nice looking UI on the computer, which can
further be projected onto a screen.  At the end of the game there will be a
list of detailed events and top 5 kills shown on the software master node.

\section{Rules of the Game} The cars will be controlled by different players.
The game will compose of a time limit and the player with the maximum kills at
the end of the game will win.  Before the game begins the players will be
able to customize their car to have different abilities.  These will further
be one of the many types of cars that will be highlighted in Section 5 -
Software Class Hierarchy.  

During the game the players will be able to deal damage to another players'
car by either shooting them with the IR gun fitted on the car or by colliding
with the other car.  The attack points for both of these will be determined by
the master node before the beginning of the game.  

\section{Design and Implementation} The cars will be built with identical
hardware before the start of the games.  All customizations will be offered to
the players.  So they can in essence pick the type of car they want and play
the game according to that.  All pre-game setup will be done on the computer
master server. 

Before the game begins, the players will all be able to select the type of car
and indicate whether they are ready to start.  This will all be given a
convenient user interface on a computer. 

Our choices for UI at the moment is a server built on a bootstrap frontend
with HTML and CSS, the backend for which will be controlled with a web server
we build.  This is justified by the extensive availability and documentation
available for libraries and frameworks like Flask, Django, Pyramid, Java
Servlet and Go’s built in libraries.  We expect to make further design
decisions here since TCP connections through the SmartFusion will have to be
sent through some existing library that interfaces with the existing network
card on the board and one that provides a programming interface for us.  Or we
will build this ourselves but we highly doubt that such a service will not
exist.  Further building our own network library will involve further
complications in the cases of latency of one connection interfering with
another and this will subsequently incur the need to build a polling mechanism
that will poll the network packet and preferably either sit waiting to to be
interrupted.

\subsection{Software and Hardware on the Cars}
All cars will be fitted with SmartFusion boards and have three input sources
\begin{enumerate}
    \item An abstracted controller that a human uses to control the car.  For
    example, if we use a wireless Gamecube controller, the receiver for the
    controller will be this abstract controller.
    \item IR sensor, this will be the sensor that senses another an attack
    from another car.
    \item XBee module that gets input from a master "server" that is
    controlling the state of the game.  So in our case the state of the game
    will be the powers of the car (i.e.  the abilities that the car is given
    by virtue of its performance so far) and the current score in the game. 
    \item Accelerometer, this tells the car whether another car has collided
    with it.  And based on the type of the car the amount of health varied
    will be different.
\end{enumerate}

And they will have some output sources
\begin{enumerate}
    \item LEDs to indicate the health of the car
    \item XBee transmissions to the master
    \item Motors that control the motion of the car
    \item IR guns that are used to attack other cars
\end{enumerate}

They are programmed to serve three purposes
\begin{enumerate}
    \item Read in input from an abstract controller and the infrared guns.
    And then update the components that correspond to its parameters, for
    example there will be 5 leds that correspond to its health.
    \item Update the components of the car to do the functions given to it by
    the controller.  This will further interface with the infrared laser that
    is meant to be a gun that shoots the other cars.
\end{enumerate}

\subsection{Software and Hardware on the Master Server}
The master "server" communicates two ways via XBee with the cars.  Every
time it gets an input, it computes the state of the game and sends this back
to either the car that it got a request from or broadcasts back to all the
cars.  The master server is composed of two parts
\begin{enumerate}
    \item A SmartFusion board that connects via XBee to the cars, and connects
    to a computer via TCP through an ethernet.
    \item A Computer that receives the state of the game via the TCP network
    from the master board and displays the state in a nice UI
\end{enumerate}

The master server will be keeping track of the game at all times.  There are
potentially two ways to do this.  
\begin{enumerate}
    \item The SmartFusion Board will receive the communication via XBee and
    will transmit this information to the computer via TCP through a connected
    ethernet cable
    \item The computer itself will have a XBee receiver connected to it and
    will constantly poll this with the help of a device driver that comes
    bundled with the XBee packages. 
\end{enumerate}

Either way the server will be displaying the state of the game on the computer
and will keep responding to the cars with response messages that can
potentially contain upgrades for the cars. 

\newpage
\section{Functional Diagram}
\begin{figure}[!htb]
\centering
\includegraphics[height=15cm]{functionaldiagram.png}
\caption{High level overview of the project components}
\end{figure}

\newpage
\section{Software Class Hierarchy}
\begin{figure}[!htb]
\centering
\includegraphics[height=8cm]{classhierarchy.png}
\caption{An Object Oriented software view of the cars}
\end{figure}

\section{Component List}
\begin{enumerate}
    \item RC cars (3 - 4)
        \begin{itemize}
            \item Driven by DC Motor
            \item Servo controls steering
        \end{itemize}
    \item Controllers (1 per car)
        \begin{itemize}
            \item Gamecube \ Nintendo 64 wireless controller and receiver
        \end{itemize}
    \item Computer (1)
    \item SmartFusion Boards (1 per car, +1 to control ‘main xbee’ )
    \item LEDs (10 per car)
        \begin{itemize}
            \item Resistors for each LED
        \end{itemize}
    \item 3D Accelerometer (1 per car)
    \item IR transmitter (1 per car)
        \begin{itemize}
            \item Tube to insert the transmitter into to narrow the field of
            view
        \end{itemize}
    \item IR receiver (1 per car)
        \begin{itemize}
            \item Tube to insert the transmitter into to narrow the field of
            view
        \end{itemize}
\end{enumerate}

\end{document}




